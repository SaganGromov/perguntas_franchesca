\documentclass[12pt]{article}
\renewcommand{\contentsname}{Sumário}
\usepackage[utf8]{inputenc}
\usepackage[T1]{fontenc}
\usepackage[portuguese]{babel}
\addto\captionsportuguese{% Replace "english" with the language you use
  \renewcommand{\contentsname}%
    {Sumário}%
}
% \usepackage{pifont}

\usepackage{xpatch}
\usepackage{textcomp}
\patchcmd{\thebibliography}{\section*{\refname}}{\section*{Bibliografia}}{}{}

\usepackage{makeidx}
\usepackage{amsmath, amsfonts,amsthm}
\usepackage{graphicx}
\usepackage[dvipsnames]{xcolor}
\usepackage{biblatex}
\usepackage[colorinlistoftodos]{todonotes}
\usepackage{lipsum} 
\providecommand{\sin}{} \renewcommand{\sin}{\hspace{2pt}\mathrm{sen}}
\newcommand{\R}{\mathbb{R}}
\newcommand{\C}{\mathbb{C}}
\usepackage{hyperref}
\hypersetup{
    colorlinks=true, %set true if you want colored links
    linktoc=all,     %set to all if you want both sections and subsections linked
    linkcolor=blue,  %choose some color if you want links to stand out
}

\newcommand\PlaceText[3]{%
\begin{tikzpicture}[remember picture,overlay]
\node[outer sep=0pt,inner sep=0pt,anchor=south west] 
  at ([xshift=#1,yshift=-#2]current page.north west) {#3};
\end{tikzpicture}%
}






\usepackage{mathrsfs}
\usepackage{faktor}

\newcommand{\bigslant}[2]{{\raisebox{.2em}{$#1$}\left/\raisebox{-.2em}{$#2$}\right.}}
\usepackage{hyperref}
\hypersetup{
    colorlinks=true, %set true if you want colored links
    linktoc=all,     %set to all if you want both sections and subsections linked
    linkcolor=black,
    citecolor = teal   %choose some color if you want links to stand out
}

\usepackage{cleveref}
\crefname{deff}{}{definitions}
\crefname{exem}{}{exemplos}
\crefname{col}{}{corolários}
\crefname{equation}{}{equações}
\creflabelformat{equation}{#2{\bf{\color{blue}(#1)}}#3}
\crefname{teorema}{}{teoremas}
\crefname{oobs}{observação}{observações}
\creflabelformat{oobs}{#2\bf{\color{black}(#1)}#3}
\crefname{lema}{}{lemas}
\creflabelformat{lema}{#2\bf{\color{black}(#1)}#3}
\crefname{proposicao}{}{proposições}
\creflabelformat{deff}{#2\bf{\color{black}(#1)}#3}
\creflabelformat{exem}{#2\bf{\color{black}(#1)}#3}
\creflabelformat{col}{#2\bf{\color{black}(#1)}#3}
\creflabelformat{proposicao}{#2{\bf\color{black}(#1)}#3}
\renewcommand{\theenumi}{\Alph{enumi}}



\definecolor{gal}{RGB}{0, 7, 111}


\makeatletter
%\patchcmd{\f@nch@head}{\rlap}{\color{black}\rlap}{}{}
%\patchcmd{\headrule}{\hrule}{\color{TealBlue}\hrule}{}{}
\patchcmd{\f@nch@foot}{\rlap}{\color{black}\rlap}{}{}
\patchcmd{\footrule}{\hrule}{\color{green}\hrule}{}{}
\makeatother




%\usepackage[shortlabels]{enumitem}

\usepackage[a4paper,bottom=0.9in,top=0.9in,left=0.3in,right=0.3in]{geometry}

\usepackage{mathtools}
\usepackage{fancyhdr}
\usepackage{lipsum}
\usepackage{enumerate}
\usepackage{enumitem}

\usepackage{lastpage}
\usepackage{graphicx}
\everymath{\displaystyle}
\newcommand{\p}{\partial}

\renewcommand{\headrulewidth}{0pt}
\renewcommand{\footrulewidth}{0pt}



\newcommand{\om}{\mathbb{M}}

\newcommand{\jps}[1]{\textcolor{blue}{#1}}
%\newcommand{\red}[1]{\textcolor{red}{#1}}
\newcommand{\pur}[1]{\textcolor{purple}{#1}}
\newcommand{\maggg}[1]{\textcolor{magenta}{#1}}

%\usepackage{color}
%\definecolor{SAEblue}{rgb}{0, .62, .91}
%\renewcommand\theequation{\red{{\arabic{equation}}}}


\makeatletter
\let\mytagform@=\tagform@
\def\tagform@#1{\maketag@@@{\bfseries\jps{(\ignorespaces#1\unskip\@@italiccorr)}}\hspace{3mm}}
\renewcommand{\eqref}[1]{\textup{\mytagform@{\ref{#1}}}}
\makeatother


%\chead{\textbf{\thepage}}
\theoremstyle{definition}
\newtheorem{deff}{Definição}
\newtheorem{lema}{Lema}
\newcommand{\ve}{\varepsilon}
\newcommand{\lnr}{\left\|}
\newcommand{\ssum}{\displaystyle\sum}
\newcommand{\rnr}{\right\|}
\newcommand{\mm}{\mathcal{M}}
%\newcommand{\R}{\mathbb{R}}
\DeclareMathOperator{\hol}{Hol}
\newtheorem*{obs*}{Notação}
%\newtheorem*{oobs}{Observação}
\newtheorem{sublema}{Sub-lema}
\renewcommand{\qedsymbol}{\rule{0.7em}{0.7em}}
\newenvironment{demm}{\smallskip \noindent{\bf Demonstração}:}
{\begin{flushright} $\qedsymbol$\end{flushright}\smallskip}
\newenvironment{dem}{\smallskip \noindent{\bf Solução}:}
{\begin{flushright} $\qedsymbol$\end{flushright}\smallskip}


\newtheoremstyle{theoremdd}% name of the style to be used
  {\topsep}% measure of space to leave above the theorem. E.g.: 3pt
  {\topsep}% measure of space to leave below the theorem. E.g.: 3pt
  {}% name of font to use in the body of the theorem
  {1pt}% measure of space to indent
  {\bfseries}% name of head font
  {}% punctuation between head and body
  { }% space after theorem head; " " = normal interword space
  {\thmname{#1} (\thmnumber{#2})\textbf{\thmnote{ (#3)}.}}


  


  
\theoremstyle{theoremdd}
\newtheorem{oobs}{Observação}
\newtheorem{teorema}{Teorema}



\makeatletter
\patchcmd{\f@nch@head}{\rlap}{\color{BlueViolet}\rlap}{}{}
%\patchcmd{\headrule}{\hrule}{\color{TealBlue}\hrule}{}{}
\patchcmd{\f@nch@foot}{\rlap}{\color{BlueViolet}\rlap}{}{}
\patchcmd{\footrule}{\hrule}{\color{green}\hrule}{}{}
\makeatother



\theoremstyle{definition}

\newtheorem{exerc}{Questão}
\usepackage{float,framed}
\setlength{\intextsep}{2pt}
\setlength{\textfloatsep}{2pt}
\newfloat{Box}{H}{0ob}
\newenvironment{Mybox}{\begin{Box}\begin{framed}\begin{exerc}}{\end{exerc}\end{framed}\end{Box}}

\newtheorem*{exercc}{Questão Extra}
\usepackage{float,framed}
\setlength{\intextsep}{2pt}
\setlength{\textfloatsep}{2pt}
\newfloat{Box}{H}{0ob}
\newenvironment{Herbox}{\begin{Box}\begin{framed}\begin{exercc}}{\end{exercc}\end{framed}\end{Box}}


\newcommand{\dd}{\mathrm{d}}
\renewcommand{\O}{\Omega}
\renewcommand{\Rn}{\mathbb{R}^n}
\renewcommand{\C}{\mathscr{C}}
\newcommand{\parent}[1]{\left( #1 \right)}
\renewcommand{\theenumi}{\textbf{(\alph{enumi})}}

