% !TeX root = main.tex

\documentclass[12pt]{article}
\renewcommand{\contentsname}{Sumário}
\usepackage[utf8]{inputenc}
\usepackage[T1]{fontenc}
\usepackage[portuguese]{babel}
\addto\captionsportuguese{% Replace "english" with the language you use
  \renewcommand{\contentsname}%
    {Sumário}%
}
% \usepackage{pifont}

\usepackage{xpatch}
\usepackage{textcomp}
\patchcmd{\thebibliography}{\section*{\refname}}{\section*{Bibliografia}}{}{}

\usepackage{makeidx}
\usepackage{amsmath, amsfonts,amsthm}
\usepackage{graphicx}
\usepackage[dvipsnames]{xcolor}
\usepackage{biblatex}
\usepackage[colorinlistoftodos]{todonotes}
\usepackage{lipsum} 
\providecommand{\sin}{} \renewcommand{\sin}{\hspace{2pt}\mathrm{sen}}
\newcommand{\R}{\mathbb{R}}
\newcommand{\C}{\mathbb{C}}
\usepackage{hyperref}
\hypersetup{
    colorlinks=true, %set true if you want colored links
    linktoc=all,     %set to all if you want both sections and subsections linked
    linkcolor=blue,  %choose some color if you want links to stand out
}

\newcommand\PlaceText[3]{%
\begin{tikzpicture}[remember picture,overlay]
\node[outer sep=0pt,inner sep=0pt,anchor=south west] 
  at ([xshift=#1,yshift=-#2]current page.north west) {#3};
\end{tikzpicture}%
}






\usepackage{mathrsfs}
\usepackage{faktor}

\newcommand{\bigslant}[2]{{\raisebox{.2em}{$#1$}\left/\raisebox{-.2em}{$#2$}\right.}}
\usepackage{hyperref}
\hypersetup{
    colorlinks=true, %set true if you want colored links
    linktoc=all,     %set to all if you want both sections and subsections linked
    linkcolor=black,
    citecolor = teal   %choose some color if you want links to stand out
}

\usepackage{cleveref}
\crefname{deff}{}{definitions}
\crefname{exem}{}{exemplos}
\crefname{col}{}{corolários}
\crefname{equation}{}{equações}
\creflabelformat{equation}{#2{\bf{\color{blue}(#1)}}#3}
\crefname{teorema}{}{teoremas}
\crefname{oobs}{observação}{observações}
\creflabelformat{oobs}{#2\bf{\color{black}(#1)}#3}
\crefname{lema}{}{lemas}
\creflabelformat{lema}{#2\bf{\color{black}(#1)}#3}
\crefname{proposicao}{}{proposições}
\creflabelformat{deff}{#2\bf{\color{black}(#1)}#3}
\creflabelformat{exem}{#2\bf{\color{black}(#1)}#3}
\creflabelformat{col}{#2\bf{\color{black}(#1)}#3}
\creflabelformat{proposicao}{#2{\bf\color{black}(#1)}#3}
\renewcommand{\theenumi}{\Alph{enumi}}



\definecolor{gal}{RGB}{0, 7, 111}


\makeatletter
%\patchcmd{\f@nch@head}{\rlap}{\color{black}\rlap}{}{}
%\patchcmd{\headrule}{\hrule}{\color{TealBlue}\hrule}{}{}
\patchcmd{\f@nch@foot}{\rlap}{\color{black}\rlap}{}{}
\patchcmd{\footrule}{\hrule}{\color{green}\hrule}{}{}
\makeatother




%\usepackage[shortlabels]{enumitem}

\usepackage[a4paper,bottom=0.9in,top=0.9in,left=0.3in,right=0.3in]{geometry}

\usepackage{mathtools}
\usepackage{fancyhdr}
\usepackage{lipsum}
\usepackage{enumerate}
\usepackage{enumitem}

\usepackage{lastpage}
\usepackage{graphicx}
\everymath{\displaystyle}
\newcommand{\p}{\partial}

\renewcommand{\headrulewidth}{0pt}
\renewcommand{\footrulewidth}{0pt}



\newcommand{\om}{\mathbb{M}}

\newcommand{\jps}[1]{\textcolor{blue}{#1}}
%\newcommand{\red}[1]{\textcolor{red}{#1}}
\newcommand{\pur}[1]{\textcolor{purple}{#1}}
\newcommand{\maggg}[1]{\textcolor{magenta}{#1}}

%\usepackage{color}
%\definecolor{SAEblue}{rgb}{0, .62, .91}
%\renewcommand\theequation{\red{{\arabic{equation}}}}


\makeatletter
\let\mytagform@=\tagform@
\def\tagform@#1{\maketag@@@{\bfseries\jps{(\ignorespaces#1\unskip\@@italiccorr)}}\hspace{3mm}}
\renewcommand{\eqref}[1]{\textup{\mytagform@{\ref{#1}}}}
\makeatother


%\chead{\textbf{\thepage}}
\theoremstyle{definition}
\newtheorem{deff}{Definição}
\newtheorem{lema}{Lema}
\newcommand{\ve}{\varepsilon}
\newcommand{\lnr}{\left\|}
\newcommand{\ssum}{\displaystyle\sum}
\newcommand{\rnr}{\right\|}
\newcommand{\mm}{\mathcal{M}}
%\newcommand{\R}{\mathbb{R}}
\DeclareMathOperator{\hol}{Hol}
\newtheorem*{obs*}{Notação}
%\newtheorem*{oobs}{Observação}
\newtheorem{sublema}{Sub-lema}
\renewcommand{\qedsymbol}{\rule{0.7em}{0.7em}}
\newenvironment{demm}{\smallskip \noindent{\bf Demonstração}:}
{\begin{flushright} $\qedsymbol$\end{flushright}\smallskip}
\newenvironment{dem}{\smallskip \noindent{\bf Solução}:}
{\begin{flushright} $\qedsymbol$\end{flushright}\smallskip}


\newtheoremstyle{theoremdd}% name of the style to be used
  {\topsep}% measure of space to leave above the theorem. E.g.: 3pt
  {\topsep}% measure of space to leave below the theorem. E.g.: 3pt
  {}% name of font to use in the body of the theorem
  {1pt}% measure of space to indent
  {\bfseries}% name of head font
  {}% punctuation between head and body
  { }% space after theorem head; " " = normal interword space
  {\thmname{#1} (\thmnumber{#2})\textbf{\thmnote{ (#3)}.}}


  


  
\theoremstyle{theoremdd}
\newtheorem{oobs}{Observação}
\newtheorem{teorema}{Teorema}



\makeatletter
\patchcmd{\f@nch@head}{\rlap}{\color{BlueViolet}\rlap}{}{}
%\patchcmd{\headrule}{\hrule}{\color{TealBlue}\hrule}{}{}
\patchcmd{\f@nch@foot}{\rlap}{\color{BlueViolet}\rlap}{}{}
\patchcmd{\footrule}{\hrule}{\color{green}\hrule}{}{}
\makeatother



\theoremstyle{definition}

\newtheorem{exerc}{Questão}
\usepackage{float,framed}
\setlength{\intextsep}{2pt}
\setlength{\textfloatsep}{2pt}
\newfloat{Box}{H}{0ob}
\newenvironment{Mybox}{\begin{Box}\begin{framed}\begin{exerc}}{\end{exerc}\end{framed}\end{Box}}

\newtheorem*{exercc}{Questão Extra}
\usepackage{float,framed}
\setlength{\intextsep}{2pt}
\setlength{\textfloatsep}{2pt}
\newfloat{Box}{H}{0ob}
\newenvironment{Herbox}{\begin{Box}\begin{framed}\begin{exercc}}{\end{exercc}\end{framed}\end{Box}}


\newcommand{\dd}{\mathrm{d}}
\renewcommand{\O}{\Omega}
\renewcommand{\Rn}{\mathbb{R}^n}
\renewcommand{\C}{\mathscr{C}}
\newcommand{\parent}[1]{\left( #1 \right)}
\renewcommand{\theenumi}{\textbf{(\alph{enumi})}}


\begin{document}

\pagestyle{fancy}
\renewcommand{\footrulewidth}{0pt}
\fancyhf{}
% \lfoot{\textbf{MATHEUS A. R. M. HORÁCIO | 231107376}}
%\rfoot{\textbf{MATRÍCULA: 17/0110923 }}
% \fancyfoot[RO]{\hspace*{2cm} \textbf{Página \thepage \ de \pageref*{LastPage}}}
%   \renewcommand\footrule{%

%  \color{BlueViolet}\noindent\makebox[\linewidth]{\rule{\paperwidth}{1pt}}
% }



\newcommand{\HRule}{\rule{\linewidth}{0.5mm}} % Defines a new command for the horizontal lines, change thickness here
% \begin{center} % Center everything on the page
 
%----------------------------------------------------------------------------------------
%	HEADING SECTIONS
%----------------------------------------------------------------------------------------



% \PlaceText{15mm}{27mm}{ \color{gal}\noindent\makebox[\linewidth]{\rule{2\paperwidth}{1.5pt}}}

% \PlaceText{11mm}{42mm}{\huge\bfseries \color{gal} Ementa do exame de qualificação em Análise}

% \PlaceText{61mm}{55mm}{\huge \bfseries \color{gal} (Segunda Área)}


% \PlaceText{15mm}{62mm}{ \color{gal}\noindent\makebox[\linewidth]{\rule{2\paperwidth}{1.5pt}}}


% \PlaceText{7mm}{20mm}{ \includegraphics[scale=0.7]{unb.eps}}


% \PlaceText{131mm}{17mm}{\bfseries \color{gal} Aluno: Matheus A. R. M. Horácio }

% \PlaceText{131mm}{23mm}{\bfseries \color{gal} Matrícula: 231107376}


% \end{center}
% \vspace{0.7cm}

\begin{itemize}
  \item \underline{\textbf{P1 Tarcísio}}
  \begin{itemize}
    \item O gráfico da função módulo pode definir uma variedade diferenciável? Pode definir uma subvariedade diferenciável imersa em $\mathbb{R}^3$? O gráfico de uma função diferenciável é sempre uma variedade completa? 
    \item Defina grupos de Lie e dê exemplos. 
    \item Enuncie o teorema da bola cabeluda e dê exemplos de aplicações.
    \item O toro admite uma métrica de curvatura zero? É possível imergir isometricamente o toro com essa métrica em $\mathbb{R}^3$? É possível dar uma métrica de curvatura seccional com sinal não nulo e constante ao toro? 
    \item Vale a recíproca do teorema Egregium? Ou seja, se duas variedades Riemannianas de dimensão $2$ têm a mesma curvatura Gaussiana, elas devem ser localmente isométricas?
    \item Quais são as isometrias da esfera? Dê um exemplo de alguma demonstração de geometria Riemanniana onde tal conhecimento possa ser aplicado. 
    \item Como é a conexão de Levi-Civita da métrica usual do espaço Euclidiano?
    \item Como é a conexão de Levi-Civita de uma superfície imersa isometricamente em $\mathbb{R}^3$? Use a conta para provar que a curvatura seccional em dimensão dois coincide com a curvatura Gaussiana.
  \end{itemize}
  \item \underline{\textbf{P2 Tarcísio}}
  \begin{itemize}
    \item Porque referenciais geodésicos são úteis? Consegue dar um exemplo de uma conta que é facilitada pelo uso de referenciais geodésicos? Consegue dar uma ideia da prova da existência de referenciais geodésicos?
    \item Como campos de Killing estão relacionado a isometrias? O que se pode dizer sobre uma variedade que tenha o número máximo possível de campos de Killing? 
    \item Quais são as geodésicas do espaço hiperbólico? Tem uma ideia de como provar a caracterização dessas?
    \item Defina o que é uma variedade de Einstein e dê exemplos. O que se pode dizer sobre variedades de Einstein em dimensão 3? Porque $\mathbb{S}^2 \times \mathbb{R}$ não admite nenhuma métrica de Einstein? Baseado na argumentação da resposta à última pergunta, dê mais exemplos de variedades que não admitem métricas de Einstein. 
    \item O que são tensores? Porque tensores são úteis? Tensores são um conceito que dependem de geometria ou podem ser definidos munidos somente da estrutura diferenciável da variedade? 
    \item Dê uma interpretação geométrica da definição de curvatura seccional. Como essa interpretação se relaciona com a maneira que Riemann definiu curvatura? Porque não faria sentido algum trabalharmos com interseções de $\mathbb{R}^3$ com os espaços tangentes $\sigma$ ao invés de $\mathbb{R}^2$?
    \item Porque não há perda de generalidade ao trabalharmos somente campos de Jacobi normais?
    \item É possível que uma variedade Riemanniana tenha curvatura seccional positiva mas não tenha pontos conjugados? Como isso se relaciona com o teorema de Hadamard? Como isso se relaciona com o parabolóide da questão C.2?
    \item De maneira geral, é possível de alguma forma caracterizar completude pelo comportamento do comprimento de certas curvas? Se sim, como?
    \item Defina a métrica produto de duas variedades Riemannianas. Note que é uma soma das métricas de cada fator. Porque se tivéssemos tentado definir como o produto ao invés da soma, não teríamos uma métrica Riemanniana? 
    \item Como relacionar a métrica do recobrimento com a métrica da variedade original? Formalize essa pergunta. 
  \end{itemize}


\end{itemize}




\end{document}
